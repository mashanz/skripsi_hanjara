%%%%%%%%%%%%%%%%%%%%%%%%%%%%%%%%%%%%%%%%%%%%%%%%%%%%%%%%%%%%%%%%%%%%%%%
% 
%%%%%%%%%%%%%%%%%%%%%%%%%%%%%%%%%%%%%%%%%%%%%%%%%%%%%%%%%%%%%%%%%%%%%%%

\chapter{\kesimpulan}
%\todo{Tambahkan kesimpulan dan saran terkait dengan perkerjaan yang dilakukan.}

%%%%%%%%%%%%%%%%%%%%%%%%%%%%%%%%%%%%%%%%%%%%%%%%%%%%%%%%%%%%%%%%%%%%%%%
% 
%%%%%%%%%%%%%%%%%%%%%%%%%%%%%%%%%%%%%%%%%%%%%%%%%%%%%%%%%%%%%%%%%%%%%%%

\section{Kesimpulan}
Meninjau hasil analisis yang di lakukan dari pengujian yang telah dilakukan, di dapat kesimpulan sebagai berikut.

\begin{enumerate}	
	\item Menggunakan lebih dari satu metode perlindungan masih memungkinkan untuk melindungi Properti intelektual perangkat keras.
	
	\item Terjadi penurunan kecepatan proses serta peningkatan kebutuhan daya pada hasil analisis simulasi di fpga.
	
	\item zero-overhead I/O bisa di lakukan dengan tambahan obfuscation-mux.
\end{enumerate}

%%%%%%%%%%%%%%%%%%%%%%%%%%%%%%%%%%%%%%%%%%%%%%%%%%%%%%%%%%%%%%%%%%%%%%%
% 
%%%%%%%%%%%%%%%%%%%%%%%%%%%%%%%%%%%%%%%%%%%%%%%%%%%%%%%%%%%%%%%%%%%%%%%

\section{Saran}
Agar Teknologi pengamanan Intelektual Properti lebih maju serta memperbaiki permasalahan yang masih ada pada penelitian ini, berikut beberapa saran dari untuk pengembangan dan penelitian selanjutnya:

\begin{enumerate}
	\item Untuk meningkatkan kecepatan akses dan mengurangi konsumsi daya dari hasil analisis level behavioral pada fpga, dibutuhkan analisis lebih lanjut pada syntesis level gate (netlist) dan level phisical (layout).
	
	\item Saat ini teknologi serta teknik perlindungan properti intelektual perangkat keras masih terbilang baru, bidang ke amanan pada IC masih minim resource serta proses manufakturing IC sendiri begitu kompleks dan luas dan spesifikasi desain setiap produk sangat rahasia. Dibutuhkan pendalaman ilmu khusus pada Level fabrikasi seperti RTL Level, Gate Level dan Layout Level.
\end{enumerate}
