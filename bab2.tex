\chapter{\babDua}
Membuat desain sebuah perangkat ic membutuhkan proses yang panjang dan sumberdaya manusia yang banyak, serta tingkat ketelitian yang tinggi. Oleh karenanya di butuhkan biaya yang tidak kecil dan waktu yang cukup lama hanya untuk membuat sebuah desain IC. Dengan kerumitan yang tinggi serta waktu yang lama dalam setiap prosesnya kadang pihak yang tak bertanggung jawab melakukan kecurangna dengan mecuri desain untuk memotong waktu dan biaya yang di butuhkan utuk produksi. sehingga menjadi masalah dalam dunia permanufakturan ic.

\section{Large Scale Integration}
\textit{Large Scale Integration} atau disingkat LSI merupakan teknologi desain IC dengan kepadatan gate sekitar xx gate. pada awalnya blablabal...

\subsection{Arus Pengembangan LSI}
\todo{isi perkembangan LSI sesua renesas web}

\subsection{Kemungkinan Serangan Desain LSI}
Terdapat banyak kemungkinan serangan dalam proses manufakturing desain LSI. Berikut beberapa contoh serangan terhadap LSI desain.

\begin{figure}
	\centering
	\includegraphics[width=1.05\textwidth]
	{diagrams/untrustSource.png}
	\caption{Clonning/Sumber Tidak Terpercaya}
	\label{fig:untrustsource}
\end{figure}

Dalam segi ini serangan dilakukan dengan cara mencuri langsung desain yang sudah siap di fabrikasi serta uji coba kebenaran. Bila pencuri mendapatkan desain yang telah di uji coba, maka pencuri tinggal langsung memperbanyak desain yang telah di curi.

\begin{figure}
	\centering
	\includegraphics[width=1.05\textwidth]
	{diagrams/reverseEngineering.png}
	\caption{RE (Reverse Engineering)}
	\label{fig:reverseengineering}
\end{figure}

Untuk serangan jenis ini, pencuri sudah pendapatkan produk dari pasar yang telah teruji, pencuri tinggal melakukan scan rangkaian kemudian mengujinya dengan datasheet. Apabila hasil scan desain produk di dapati rangkaian yang konkrit/jelas dan rangkaian tersebut telah teruji sesuai datasheet. Maka pencuri tinggal melakukan fabrikasi.

\subsection{Mengatasi Serangan terhadap Desain LSI}
\todo{jelasin dari buku dan paper gimana cara mengatasinya}

\section{Teknik Proteksi}
\todo{Jelasin yg gue kerjain, terus kasih kelebihannya, agak mikir sih, but its simple may be}

\subsection{DSP (Digital Signal Processing)}
\todo{dsp penjelasan}

\subsection{Polimorphisme}
\todo{polymorph penjelasan}

\section{Peralatan dan Teknologi}
\todo{isi sendiri}

\subsection{Verilog HDL}
\todo{isi sendiri}

\subsection{Yosys}
\todo{isi sendiri}

\subsection{ISE Design Suit}
\todo{isi sendiri}

\subsection{Electric VLSI}
\todo{isi sendiri}

\subsection{FPGA Elbert V2 Board}
\todo{isi sendiri}