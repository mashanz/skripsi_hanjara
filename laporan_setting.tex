% Judul laporan. 
\var{\judul}{PERANCANGAN DAN SIMULASI\\
	PERLINDUNGAN PROPERTI INTELEKTUAL\\
	MENGGUNAKAN ALGORITME \textit{OBFUSCATION} FILTER DIGITAL}

% Tulis kembali judul laporan, kali ini akan diubah menjadi huruf kapital
\Var{\Judul}{PERANCANGAN DAN SIMULASI
	PERLINDUNGAN PROPERTI INTELEKTUAL
	MENGGUNAKAN ALGORITME \textit{OBFUSCATION} FILTER DIGITAL}

\Var{\Judulen}{DESIGN AND SIMULATION OF INTELECTUAL PROPERTIES PROTECTION USING DIGITAL FILTER ALGORITHM OBFUSCATION}

% Tulis kembali judul laporan namun dengan bahasa Ingris
\var{\judulInggris}{Unknown Title for Final Report/Thesis/Disertation}

% Tipe laporan, dapat berisi Skripsi, Tugas Akhir, Thesis, atau Disertasi
\var{\type}{TUGAS AKHIR}

% Tulis kembali tipe laporan, kali ini akan diubah menjadi huruf kapital
\Var{\Type}{TUGAS AKHIR}

% Tulis nama penulis 
\var{\penulis}{Hanjara Cahya Adhyatma}

% Tulis kembali nama penulis, kali ini akan diubah menjadi huruf kapital
\Var{\Penulis}{Hanjara Cahya Adhyatma}

% Tulis NPM penulis
\var{\npm}{1104131113}
\var{\nim}{1104131113}
\var{\alamat}{Komplek BPI Blok E1 No. 16 RT 05 RW 06 Kabupaten Pandeglang, Banten}
\var{\email}{adhyatma.han@gmail.com}
\var{\tlp}{+6285201740588}
% Tuliskan Fakultas dimana penulis berada
\Var{\Fakultas}{Teknik Elektro}
\var{\fakultas}{Tekniks Elektos}

% Tuliskan Program Studi yang diambil penulis
\Var{\Program}{S1 Sistem Komputer}
\var{\program}{S1 Sistems Komputers}

% Tuliskan tahun publikasi laporan
\Var{\bulanTahun}{September 2017}
\Var{\Tahun}{2017}
% Tuliskan gelar yang akan diperoleh dengan menyerahkan laporan ini
\var{\gelar}{Sarjana Teknik}

% Tuliskan tanggal pengesahan laporan, waktu dimana laporan diserahkan ke 
% penguji/sekretariat
\var{\tanggalPengesahan}{XX September 2017} 

% Tuliskan tanggal keputusan sidang dikeluarkan dan penulis dinyatakan 
% lulus/tidak lulus
\var{\tanggalLulus}{XX September 2017}

\var{\pembimbing}{Prof. XXXX}
\var{\pembimbingSatu}{Surya Michrandi Nasution, S.T., M.T.}
\var{\pembimbingDua}{Fairuz Azmi, S.T., M.T.}

\var{\saya}{Hanjara Cahya Adhyatma}

\Var{\kataPengantar}{Kata Pengantar}
\Var{\babSatu}{Pendahuluan}
\Var{\babDua}{Tinjauan Pustaka}
\Var{\babTiga}{Desain dan Simulasi}
\Var{\babEmpat}{Pengujian dan Analisis}
\Var{\kesimpulan}{Kesimpulan dan Saran}
